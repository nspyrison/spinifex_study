\documentclass[]{article}
\usepackage{lmodern}
\usepackage{amssymb,amsmath}
\usepackage{ifxetex,ifluatex}
\usepackage{fixltx2e} % provides \textsubscript
\ifnum 0\ifxetex 1\fi\ifluatex 1\fi=0 % if pdftex
  \usepackage[T1]{fontenc}
  \usepackage[utf8]{inputenc}
\else % if luatex or xelatex
  \ifxetex
    \usepackage{mathspec}
  \else
    \usepackage{fontspec}
  \fi
  \defaultfontfeatures{Ligatures=TeX,Scale=MatchLowercase}
\fi
% use upquote if available, for straight quotes in verbatim environments
\IfFileExists{upquote.sty}{\usepackage{upquote}}{}
% use microtype if available
\IfFileExists{microtype.sty}{%
\usepackage{microtype}
\UseMicrotypeSet[protrusion]{basicmath} % disable protrusion for tt fonts
}{}
\usepackage[margin=1in]{geometry}
\usepackage{hyperref}
\hypersetup{unicode=true,
            pdftitle={The effects of interaction on linear projections of multivariate data},
            pdfauthor={Nick, Di, Kim},
            pdfkeywords={exploratory data analysis, projection pursuit, high dimensional data,
data visualization, cluster analysis, dimension reduction, statistical
graphics, data science, user study, between users,},
            pdfborder={0 0 0},
            breaklinks=true}
\urlstyle{same}  % don't use monospace font for urls
\usepackage{graphicx,grffile}
\makeatletter
\def\maxwidth{\ifdim\Gin@nat@width>\linewidth\linewidth\else\Gin@nat@width\fi}
\def\maxheight{\ifdim\Gin@nat@height>\textheight\textheight\else\Gin@nat@height\fi}
\makeatother
% Scale images if necessary, so that they will not overflow the page
% margins by default, and it is still possible to overwrite the defaults
% using explicit options in \includegraphics[width, height, ...]{}
\setkeys{Gin}{width=\maxwidth,height=\maxheight,keepaspectratio}
\IfFileExists{parskip.sty}{%
\usepackage{parskip}
}{% else
\setlength{\parindent}{0pt}
\setlength{\parskip}{6pt plus 2pt minus 1pt}
}
\setlength{\emergencystretch}{3em}  % prevent overfull lines
\providecommand{\tightlist}{%
  \setlength{\itemsep}{0pt}\setlength{\parskip}{0pt}}
\setcounter{secnumdepth}{0}
% Redefines (sub)paragraphs to behave more like sections
\ifx\paragraph\undefined\else
\let\oldparagraph\paragraph
\renewcommand{\paragraph}[1]{\oldparagraph{#1}\mbox{}}
\fi
\ifx\subparagraph\undefined\else
\let\oldsubparagraph\subparagraph
\renewcommand{\subparagraph}[1]{\oldsubparagraph{#1}\mbox{}}
\fi

%%% Use protect on footnotes to avoid problems with footnotes in titles
\let\rmarkdownfootnote\footnote%
\def\footnote{\protect\rmarkdownfootnote}

%%% Change title format to be more compact
\usepackage{titling}

% Create subtitle command for use in maketitle
\providecommand{\subtitle}[1]{
  \posttitle{
    \begin{center}\large#1\end{center}
    }
}

\setlength{\droptitle}{-2em}

  \title{The effects of interaction on linear projections of multivariate data}
    \pretitle{\vspace{\droptitle}\centering\huge}
  \posttitle{\par}
    \author{Nick, Di, Kim}
    \preauthor{\centering\large\emph}
  \postauthor{\par}
    \date{}
    \predate{}\postdate{}
  

\begin{document}
\maketitle
\begin{abstract}
abstract text here
\end{abstract}

\hypertarget{introduction}{%
\section{Introduction}\label{introduction}}

\hypertarget{hypothesis}{%
\section{Hypothesis}\label{hypothesis}}

Does the finer control afforded by the manual tour improve the ability
of the analyst to understand the importance of variables contributing to
the structure?

\hypertarget{sec:results}{%
\section{Experimental design}\label{sec:results}}

\hypertarget{participant-population}{%
\subsection{Participant population}\label{participant-population}}

A sample of convience was taken from postgraduate students in the
department of econometrics and business statistics at Monash University,
based in Melbourne, Australia.

\hypertarget{sec:vis_methods}{%
\subsection{Factors}\label{sec:vis_methods}}

Each participant was randomly split into one of 3 even factor groups.
The first group saw only a single static linar projection, that of first
two principal componets of the data. The second group watched a 30
second loop of Grand tour. The remaining group was allowed to control an
interactive manual tour for their duration.

\hypertarget{block-treatments}{%
\subsection{Block treatments}\label{block-treatments}}

Each participant performed each of 4 block treatments in random order.
The blocks consisted of determing the the dimensionality of the dataset,
\texttt{p}, the number of the clusters, \texttt{n}, the number of
important variables, \texttt{d}, and if there existed an significant
covariance, \texttt{s}.

\hypertarget{randomization-replication}{%
\subsection{Randomization \&
replication}\label{randomization-replication}}

Paricipants were randomly assigned to one of 3 factors deciding which
visual methed they recieved. The blocks were performed in a random order
for each participant. Within each block, pacticipants performed 4
replications, answering the block question for each of the 4 datasets in
a random order before proceeding to the next block.

\hypertarget{response-measures}{%
\subsection{Response \& measures}\label{response-measures}}

Each block was introduced and demonstrated directly preceeding each
block. During this introductory question each participant was shown the
visual for their factor with a writen description of the block and how
to descern it with the same toy data set. Participants recieved exactly
2 minutes to study/explore each repitition's projection before answering
a question regarding it. Answers came in the form of a numeric input for
three blocks - namely, dimensionality, clusters, and imporant variables
(\texttt{p}, \texttt{n}, and \texttt{d} respectivly). For the remaining
block, covariance \texttt{s}, a checkmark box was provided for each
varible. Pacticipants we instructed to mark all variables, if any, that
were highly correlated. None of the data sets contained more than a
single group of highly-correlated variables.

After responses for each block were collected, paricipants were given a
short survey of 7 subjective questions on a 9-point Likert scale. These
questions covered familiarity and expertise with multivariate data, its
visualization, as well as, ease of use, understandability, confidence,
and likely hood to recomend their factors visualization.

Eye tracking devices were also used to follow gaze of participants as
the study, including the survey was conducted.

\hypertarget{sec:results}{%
\subsection{Experimental results}\label{sec:results}}

\hypertarget{discussion}{%
\section{Discussion}\label{discussion}}

\hypertarget{acknowledgments}{%
\section{Acknowledgments}\label{acknowledgments}}

\hypertarget{bibliography}{%
\section{Bibliography}\label{bibliography}}


\end{document}
